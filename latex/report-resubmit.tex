\documentclass[dvipdfmx]{jsarticle}
\usepackage[T1]{fontenc}
\usepackage[dvipdfmx]{hyperref}
\usepackage{lmodern}
\usepackage{latexsym}
\usepackage{amsfonts}
\usepackage{amssymb}
\usepackage{mathtools}
\usepackage{amsthm}
\usepackage{multirow}
\usepackage{graphicx}
\usepackage{wrapfig}
\usepackage{here}
\usepackage{float}
\usepackage{ascmac}
\usepackage{url}

\title{統計学1 中間課題 答案(再提出)}
\author{文理学部 情報科学科\\5419045 高林 秀}
\date{\today}

\begin{document}

\maketitle

\begin{abstract}
    本稿は、後期総合教育科目である統計学1の中間課題として与えられた、健康診断データの分析に関するレポートである。\par 
    平均や、標準偏差といった各種指標数値の計算にはPythonのライブラリであるPandasを用いて計算を行った。また、グラフの描画にはmatplotlibを使用した。\par 
    はじめに、各数値データ(年齢、身長、体重、最大血圧、最小血圧)に関して、それぞれ平均や標準偏差、度数分布表、ヒストグラムを作成し、元データ全体がどのような分布になっているかについて考察した。 
    その後、各数値(量的)データと質的データ(以降は「血圧判定」「心電図判定」を指す)の関連性を調べるため、相関比を算出し、結果から導かれることを考察した。
    その後、質的データ同士の関連性をクラメールの関連係数を用いて考察した。\par 
    最後に、算出した各種指標値の結果から、どの属性が質的データに影響を与えているのかを考察した。
\end{abstract}
\tableofcontents

\section{各数値データの概要}
\section{各数値データの相関比}
\section{各属性のクラメールの関連係数}
\section*{付録}
    \subsection*{計算機環境}
    \subsection*{図表一覧}
    \subsection*{ソースコード}
\end{document}